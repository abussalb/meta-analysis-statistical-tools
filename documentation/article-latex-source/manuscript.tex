\documentclass[12pt,a4paper,english]{article}


% adding the line below for Multifile document support with LatexTools Sublime package 
%!TEX root = manuscript.tex


\usepackage[utf8]{inputenc}
\usepackage[english]{babel}

% Quotes 
\usepackage[square]{natbib}
\bibliographystyle{abbrvnat}

% For hyperlinks in the pdf 
\usepackage{hyperref}

% Glossaries
\usepackage[acronym]{glossaries}

% Font Helvetica
\renewcommand{\familydefault}{\sfdefault}
\usepackage[T1]{fontenc}

% Margins
\usepackage{geometry}
 \geometry{
 a4paper,
 total={170mm,257mm},
 left=20mm,
 top=20mm,
 }

% Linespace
\linespread{1.5}

% For pictures in the pdf
\usepackage{graphicx}

% For tables
\usepackage{multirow}
\usepackage{booktabs}
\usepackage{threeparttable}

% For ref with figure, table or equation before the number
\usepackage{cleveref}

% landscape page
\usepackage{lscape}

% for argmin and argmax
\usepackage{amsmath}
\DeclareMathOperator*{\argmin}{argmin}

% Glossary 
\usepackage[acronym]{glossaries}
\makeglossaries


% This is for me to comment

% Not using the pdfcomment package but it is an interesting one  
%\usepackage[author={Louis Mayaud}]{pdfcomment}

% Select what to do with todonotes: 
% \usepackage[disable]{todonotes} % notes not showed
\usepackage[draft]{todonotes}   % notes showed

% Select what to do with command \comment:  
% \newcommand{\comment}[1]{}  %comment not showed
\newcommand{\comment}[1]
{\par {\bfseries \color{blue} #1 \par}} %comment showed

%\makeglossaries

\newacronym{adhd}{ADHD}{Attention deficit/hyperactivity disorder}
\newacronym{nfb}{NFB}{Neurofeedback}
\newacronym{eeg}{EEG}{electroencephalogram}
\newacronym{smr}{SMR}{sensorimotor rhythm}
\newacronym{mri}{MRI}{magnetic resonance imagery}
\newacronym{mph}{MPH}{methylphenidate}
\newacronym{tbr}{TBR}{theta beta ratio}
\newacronym{scp}{SCP}{slow cortical potential}
\newacronym{erp}{ERP}{event-related potential}
\newacronym{rct}{RCT}{randomized controlled trial}
\newacronym{hkd}{HKD}{hyperkinetic disorder}
\newacronym{es}{ES}{effect size}
\newacronym{se}{SE}{summary effect}
\newacronym{irb}{IRB}{Institutional Review Board}
\newacronym{ols}{OLS}{Ordinary Least Squares}
\newacronym{wls}{WLS}{Weighted Least Squares}
\newacronym{wrss}{WRSS}{Weighted Residual Sum of Squares}
\newacronym{lasso}{LASSO}{Least Absolute Shrinkage and Selection Operator}
\newacronym{pblind}{Pblind}{probably blind}
\newacronym{mprox}{MProx}{most proximal}
\newacronym{eog}{EOG}{electro-oculogram}
\newacronym{mse}{MSE}{Mean Square Error}
\newacronym{tova}{TOVA}{Test of Variables of Attention}
\newacronym{cpt}{CPT}{Continuous Performance Test}
\newacronym{sart}{SART}{Sustained Attention to Response Task}
\newacronym{fmri}{fMRI}{functional Magnetic Resonance Imaging}
\newacronym{pet}{PET}{Positron Emission Tomography}
\newacronym{saob}{SAOB}{systematic analysis of biases}
\newacronym{agcl}{AgCl}{Silver Chloride}
\newacronym{au}{Au}{Gold}
\newacronym{emg}{EMG}{Electromyogram}


\begin{document}

% adding the line below for Multifile document support with LatexTools Sublime package 
%!TEX root = manuscript.tex

% Title

\title{Effectiveness of Neurofeedback in children with ADHD: design choices that really matter} % Article title
\maketitle
\noindent Aurore Bussalb$^a$, PhD student, Marco Congedo$^b$, PhD, Quentin Barth\'elemy$^a$, PhD, David Ojeda$^a$, PhD, 
Eric Acquaviva$^c$, MD, PhD, Richard Delorme$^c$, MD, PhD, Louis Mayaud$^a$, D.Phil. 
\smallbreak
\noindent $^a$: Mensia Technolgies, Paris, France \\
\noindent $^b$: GIPSA-Lab, CNRS, University Grenoble Alpes, National Polytechnic Institute, Grenoble, France \\
\noindent $^c$: Hôpital Robert Debré, Paris, France \\ 
\smallbreak
\noindent\textbf{Full postal address} \\
Mensia Technologies \\
Plateforme d'innovation Boucicaut \\
130 rue de Lourmel \\
75015 Paris - France \\
Aurore Bussalb's e-mail: aurore.bussalb@mensiatech.com \\
Louis Mayaud's e-mail: lm@mensiatech.com 
\smallbreak
A. Bussalb, Q. Barth\'elemy, D. Ojeda, and L. Mayaud work for Mensia Technologies.
M. Congedo served as an advisor for Mensia Technologies when this work was conducted. 
\smallbreak
This research was conducted as part of a PhD thesis funded by Mensia Technologies and the Association Nationale 
Recherche et Technologie (ANRT) as well as the EU H2020 NEWROFEED grant 684809.

\noindent Number of words: 5701 \\
\noindent Number of figures: 4 


\clearpage

\begin{abstract}
% adding the line below for Multifile document support with LatexTools Sublime package 
%!TEX root = manuscript.tex

% Abstract

\noindent Numerous trials and several meta-analysis have been published on the efficacy of Neurofeedback (NFB) applied
to Attention Deficit Hyperactivity Disorder (ADHD) in children and adolescents with inconsistent findings.

This work replicated the latest meta-analysis on that topic (2016) and, by doing so, benchmarked methodological choices
originally made and later challenged.

Furthermore, the meta-analysis was updated including two recently published randomized control trials.

This process revealed the heterogeneity of studies included in past meta-analysis, which questions the reliability of
their results.

The analysis was therefore completed with a novel method: the systematic analysis of biases (SAOB) that takes advantage
of studies technical and methodological heterogeneity rather than suffering from it.

The SAOB was performed on k = 31 studies meeting the same inclusion criteria as for the update of the meta-analysis (but the requirement for a control arm).

The update of the most recent meta-analysis with two new publications confirmed the results originally obtained: effect
sizes were significant when clinical scales of ADHD were rated by parents (p-value = 0.0017) but not when teachers did
(considered as probably blind, p-value = 0.14).

Also, significant improvements were confirmed for the subset of studies meeting the definition of "standard NFB
protocols" even when clinical outcomes were observed by probably blind raters (p-value = 0.043, k = 4 studies).

The SAOB identified 3 elements that might have an impact on NFB efficacy: first, a more intensive treatment was
associated with higher efficacy; second, high-end EEG systems improved the effectiveness of NFB in ADHD; third, the
person assessing the symptoms changes during trials had an impact on results: teachers seemed to score less improvement.

In conclusion, more than replicating previous findings, we introduced here a new way to look into the heterogeneity of
clinical trials.  

% frontiers: 350

\vskip 0.2in
\noindent keywords: ADHD, Neurofeedback, influencing factors, analysis of bias, linear regression, decision tree, meta-analysis.
\end{abstract}

% adding the line below for Multifile document support with LatexTools Sublime package 
%!TEX root = manuscript.tex

% Introduction

\section{Introduction} 

\gls{adhd} is a common psychiatric disorder of childhood characterized by impaired attention and/or hyperactivity/impulsivity, 
symptoms which may persist in adulthood with clinical significance which makes \gls{adhd} a life-long problem for many patients 
\citep{Faraone2006}. The prevalence of \gls{adhd} is about 5\% in school-aged children yielding to an estimated 2.5 millions of 
children in Europe \citep{DSM-5}. \gls{adhd} has an impact on the children well being because many of them may have low self 
esteem \citep{Shaw2005} and underachieve at school \citep{Barry2002} but parents are also affected by this situation \citep{Harpin2005}: 
they are often stigmatized due to the fact that for many, the behavior of children with \gls{adhd} is solely explained by bad 
parenting. Besides \gls{adhd} has a financial cost: it is estimated at between \$12,005 and \$17,458 per individual annually \citep{Pelham2007}. 

The diagnosis of \gls{adhd} primarily relies on questionnaire-based clinical evaluation \citep{DSM-5}, which can be supported 
with objective assessment metrics of executive function such as the \gls{tova} \citep{Forbes1998}, the \gls{cpt} \citep{Barkley1991} 
and the \gls{sart} \citep{Robertson1997}. On the contrary, objective markers of brain function using \gls{eeg}, \gls{fmri}, or \gls{pet}
could not successfully improve diagnosis \citep{Neba} at the individual level but proved significantly different on the 
population. More specifically, these studies allowed to identify specific neurophysiological phenotypes of \gls{adhd}: 
this was particularly reported with \gls{eeg} recordings \citep{loo2017}. For instance, \gls{adhd} patients were found to show 
an increase in theta waves (4-8Hz) in the frontal area whereas there are less beta waves (12-32Hz) and \gls{smr} (13-15Hz) 
in the central area \citep{Monastra2005, Matouvsek1984, Janzen1995}.  
 
Among all existing treatments, the most widely used is the psychostimulants, e.g. \gls{mph}, which has been proven to be 
efficacious \citep{Taylor2014, Storebo2015}. However, the long-term effects when taking psychostimulants
are not established: it seems that the decrease of \gls{adhd} symptoms does not persist when the patient stops the treatment
\citep{DuPaul1998, Swanson2001, Jensen1999}. Moreover, \gls{adhd} children under medication commonly suffer from side effects
such as loss of appetite and sleep problems but no serious adverse events have been reported \citep{Storebo2015, Cooper2011}. 
These drawbacks make some parents and clinicians reluctant to choose such medications, so they turn to drug-free
treatment options such as dietary changes \citep{Belanger2009} and behavioral therapy which are in most of cases less efficient \citep{Sonuga-Barke2013}.

\gls{nfb} is a noninvasive technique based on behavioral therapy that aims to reduce the \gls{adhd} symptoms \citep{Arns2015, Steffert2010}.
It is a self-paced brain neuromodulation technique that represents one's brain activity in real-time using auditory or 
visual modulations, on which learning paradigms can be applied such as operant conditioning or voluntary control.
To deliver this intervention, neurophysiological time series must be recorded and analyzed in real-time and implemented in serious games 
leveraging learning paradigms. To that effect, recorded brain signals are analyzed to extract a real time representations of the activity 
of a population of neurons involved in attentional networks to which learning paradigms are applied, which is translated into a visual 
or auditory cues. The sensory feedback constitutes the rewards mechanism that promotes learning using a well-known operant conditioning protocol. 
The operant conditioning principle will enable the child to repeat more and more easily this task and thanks to the natural neuronal plasticity,
a neuronal reorganization is observed \citep{VanDoren2017}. 

In case of \gls{adhd}, several \gls{nfb} protocols have been proposed and investigated to decrease the symptoms: 
\begin{itemize}
	\item protocols based on frequency band training: a child can be asked to enhance his \gls{smr} 
	while suppressing theta or beta \citep{Lubar1976}, or he can have to enhance beta
	while suppressing theta (this scenario is known as \gls{tbr}) \citep{Arns2013};
	\item protocol based on the \glspl{scp} training which consists in the regulation of cortical excitation 
	thresholds by focusing on activity generated by external cues 
	(similar to \glspl{erp}) \citep{Heinrich2004, Banaschewski2007}; 
	\item protocol based on \glspl{erp} (P300) \citep{Fouillen2017}: \gls{adhd} children have a reduced P300 
	amplitude so it can be considered as a specific neurophysiological marker of selective attention. 
\end{itemize} 

Shortly after the discovery of the brain's electric activity by Hans Berger in 1924, \citet{Durup1935} proved it could be voluntarily modulated. 
The first indication of its therapeutic potential came forty years later when \citet{Sterman1974} serendipitously found the training of \gls{smr} 
activity to reduce the incidence of epileptic crisis in kerozen-exposed cats. The technique, then known as \gls{nfb} quickly became investigated in 
various fields of neuropsychiatry including, most notably, \gls{adhd} and resulting in a relatively large body of scientific literature 
\citep{Lubar1976, Rossiter1995, Linden1996, Maurizio2014}. Subsequently, its efficacy on the core symptoms of \gls{adhd} (inattention, hyperactivity 
and impulsivity) has been subject to several meta-analytic studies \citep{Loo2005, Lofthouse2012, Arns2009, Micoulaud2014, Sonuga-Barke2013}. 

%\comment{Not sure how, this paragraph is structured. I would take the message we carved for your SOFTAL presentation: 1. What is it? 2. Prevalence/incidence, 3. Consequences (social and financial),
% 4. diagnosis methods (introduce limitation of biomarkers), 
% 5. Existing treatment and their limitation. Then only introduce NFB with its origins. You will find interesting material and references in the CER document. }

The most recent meta-analysis solely on the efficacy of \gls{nfb} has been conducted by \citet{Cortese2016} in 
which 13 studies are included. Although only \glspl{rct} are selected, the authors of 
this meta-analysis have made some choices which have been debated by the community in particular by 
\citet{Micoulaud2016} who criticized the use of an uncommon behavioral scale provided by \citet{Steiner2014}
for the teachers' assessments and the inclusion of a pilot study carried out by \citet{Arnold2014} in the meta-analysis. 

Because of the publication of new researches meeting \citeauthor{Cortese2016}'s inclusion criteria, we decided to update his work and take 
the opportunity to investigate some choices that later proved controversial. Eventually, we extended the analysis with a \gls{saob} taking 
advantage of studies technical and methodological high heterogeneity rather than suffering from it. Indeed, the \gls{nfb} 
domain is characterized by a clinical literature that is tremendously heterogeneous: studies differ on a methodological 
point of view (randomization and presence of a blind assessor for instance), but also on the \gls{nfb} implementation (number of sessions,
session and treatment length and type of protocol for example) and on the acquisition and pre-processing of the \gls{eeg}. Since we supposed that the 
methodological and technical choices made by authors may lead to various \gls{nfb} results, we propose here to identify which of the factors 
independently influence the reported \gls{es} thanks to adequate statistical tools.


%\comment{This last paragraph is not very clear for me either even though I know what you did and why. I would clearly split your point as follow:
% 1. we wanted to replicate Cortese's work in the light of recently published clinical work meeting his inclusion/selection criteria, 2. this was a
% good opportunity to study the sensitivity of his methodological choices that were questioned, and finally 3.
%* given the large heterogeneity of the studies included in the analysis (detail the heterogeneity and give some specifics in the intro) 
%we decided to offer a new framework for the analysis so as to benefit from it - rather than it constitutiong a major limitation of the work.  }

% number of words: 821






% adding the line below for Multifile document support with LatexTools Sublime package 
%!TEX root = manuscript.tex

% Materials and Methods

\section{Materials and Methods}

\subsection{Studies selection}

Search terms described in Supplemental Material were entered in Pubmed and studies included in previous meta-analysis were identified. Among these studies, those that
satisfied each of these points were selected:
\begin{itemize}
	\item assess \gls{nfb} efficacy; 
	\item subjects must be diagnosed \gls{adhd} based on DSM-IV \citep{DSM-4}, DSM-V \citep{DSM-5}, ICD-10 \citep{ICD101993} 
	criteria or according to an expert psychiatrist; 
	\item language is English, German or French;
	\item include at least 8 subjects in each group;
	\item subjects must be younger than 25 years old;
	\item provide enough data to compute required metrics for the following analysis.
\end{itemize} 
The studies satisfying all these points were included in the \gls{saob}, then in order to replicate and 
update \citeauthor{Cortese2016} meta-analysis, we apply the inclusion criteria of this meta-analysis to the found studies. 

\subsection{Outcome definition} 

In included studies, the severity of \gls{adhd} symptoms have been assessed by parents and, when available, by teachers. \citet{Cortese2016} 
and \citet{Micoulaud2014} defined parents as \gls{mprox} raters who are not blind to the treatment of their child, as opposed to 
teachers who are considered as \gls{pblind} raters. This distinction is meant to assess the amplitude of the placebo effect where 
it is hypothesized that teachers who are presumed more blind to the intervention are less influenced in their assessment by their perception of it. 
Efficacy of \gls{nfb} was given for the following outcomes on clinical scales when available: inattention, 
hyperactivity/impulsivity and total scores. The factors analysis was performed using the total score only.

\subsection{Meta-analysis}

Meta-analysis are typically used to aggregate results from different clinical investigations and offer a consolidated 
state of the evidence. To aggregate results from different studies, it is necessary to assume some level of homogeneity 
in their design relative to: the inclusion criteria, the intervention, the presence, and type of control.
Because studies occasionally use slight variations of a clinical scales and because the populations and 
control groups might vary in their nature, the scores are typically standardized before being pooled. 
The between \gls{es} is one of such standardized metrics, which we implemented in this paper as described 
in the Supplemental Materials \citep{add exact reference here}. The work was carried with an open-source 
Python package developed for this work that offers a more transparent approach to the choice or parameters 
and increases replicability. This package was benchmark against RevMan v5.3 \citep{RevMan}
by replicating \citet{Cortese2016}'s work and obtaining the same results. The code is made fully available 
on a GitHub repository \cite{add exact reference here} including raw data for everyone to review its implementation, update it, and 
use it for different projects. 
 
Before updating the \citet{Cortese2016} work with recently published clinical work meeting his inclusion criteria 
\citep{Strehl2017, Baumeister2016}, we decided to run a sensitivity analysis investigating choices that later 
proved controversial \citep{Micoulaud2016}. Altogether, the changes investigated in our update included the following:
\begin{itemize}
\item the \gls{es} of \citeauthor{Arnold2014} study is computed from the post-test clinical values taken after the completion of the 40 sessions 
contrary to \citet{Cortese2016} who used the results after 12 sessions of \gls{nfb} because final values were not available;
\item the \gls{es} computed from teachers' assessments for \citet{Steiner2014} rely on the BOSS Classroom Observation \citep{Shapiro2010} whereas 
another scale more often used \citep{Christiansen2014, Bluschke2016} and which is the revision of the Conners Rating Scale Revised \citep{Conners1998} 
and whose reliability has been studied \citep{Collett2003} was provided. Thus we decided to compute the \gls{es} based on the results from the Conners.  
\end{itemize} 

Eventually, the new studies meeting the inclusion criteria defined by \citeauthor{Cortese2016} were added to the replication of the meta-analysis. 

As initially suggested we performed two subgroups analysis: first, \gls{se}, the weighted average of all the \gls{es}, was calculated with only studies following 
standard protocol as defined by \citet{Arns2014} and second with studies whose participants take low-dose or no medication during the trial. 
These analysis were performed with the choices described above. 

\subsection{Detect factors influencing the Neurofeedback}

While revisiting the work carried on meta-analysis, it became apparent that the studies pooled together where highly heterogeneous 
in terms of methodological and practical implementation. For instance, all \gls{nfb} interventions were pooled together irrelevant to the 
quality of the acquisition, the level excreted on real time data quality, and the trained neuromarker. 
Likewise, the methodological implementations varied significantly, requiring the 'subgroup' analysis (low drug, standard protocols) 
that are somewhat arbitrary. To circumvent these limitations, we implemented a novel approach taking advantage of the studies heterogeneity 
rather than suffering from it. In this setting, the within-\gls{es} of each intervention is considered as a dependent variable that
methodological and technical biases try to explain using standard statistical tools. The results of such analysis should enable us to identify 
known methodological biases (e.g.\ blind assessments negatively associates with \gls{es}) 
and possibly technical factors (e.g.\ a good control on real time data quality influences positively the treatment outcome). 

\subsubsection{Identify and pre-process factors}

We classified the factors influencing the efficacy of \gls{nfb} in 5 categories: methodological, technical, characteristics of the included
population, representative of the quality of acquisition and of the signal. 
Factors were chosen based on what was typically reported in the literature and presumed to influence \gls{es} and categorized as follow:

\begin{itemize}
\item \emph{signal quality}: the ocular artifacts correction and the artifact correction based on amplitude; 
\item \emph{population}: the psychostimulants intake during \gls{nfb} treatment and the age bounds of children;
\item \emph{methodological biases}: the presence of a control group, the blinding of assessors, 
the randomization of subjects, and the approval by an \gls{irb};
\item \emph{\gls{nfb} implementation}: the protocol used (\gls{scp}, \gls{smr}, 
theta up, beta up in frontal areas, theta down), the presence of a transfer phase during \gls{nfb} training, the possibility to train at home 
or at school with a transfer card reminding of the \gls{nfb} session, 
the type of thresholding reward, the number of \gls{nfb} sessions, the sessions frequency during a week, the session length and the treatment length;
\item \emph{quality of acquisition}: the presence of one or more active electrode and the \gls{eeg} quality. 
This last factor was an indicator between 1 and 3: if \gls{eeg} was recorded and processed in poor conditions then the indicator would be 1. 
Besides, if the article didn't precise the recording conditions, the factor would be set to 1. To get an indicator bigger than 1, several 
points had to be satisfied:
\begin{itemize}
  \item \emph{the type of electrodes used}: \gls{agcl}/Gel and \gls{au}/Gel;
  \item \emph{check of the electrode contact quality trough the amplifier impedance acquisition mode}: inter-electrode impedance must be smaller than $40k\Omega$;  
  \item \emph{the amplifier used}: those that are conformed to European norms (such as Thera Prax\textregistered, by NeuroConn\copyright,
	Germany \citep{NeuroCare} and Eemagine\copyright, Germany \citep{Eemagine}) are preferable or whose reliability is known.
\end{itemize}
\end{itemize}

We provide in the Github repository \ref{} the raw data extracted from the publications. To prevent any bias, the names of these factors
were hidden during the entire analysis so that the data scientist (AB) was fully blind to them. The variable names were only revealed once the data 
analysis and results were accepted as valid: this included choice of variable normalization and validation of model hypothesis as detailed below.

The pre-processing of factors for the analysis included the following steps: factors for which there were too many missing observations, 
arbitrarily set to more than 20\% of the total of observations, were removed from the analysis. Furthermore, if a factor had more than 
80\% similar observations it was removed as well. Categorical variables were coded as dummies meaning that the presence of the factor was represented by a 1 
and its absence by 0. All variables were standardized, except when the decision tree was performed. 

%\comment{be consistent with the use of time in your manuscript. I would use past tense for introduction and  materials and methods. Present for the rest.}

\subsubsection{Associate independent factors to effect sizes}

To compute this \gls{es}, means of total \gls{adhd} score given by parents and teachers were used. Besides, in case studies provided results 
for more than one behavioral scale, \gls{es} were computed for each one as 

\begin{equation}
\label{eq:factors_effect_size_within_subject}
\text{ES} = \frac{M_{\text{post},T} - M_{\text{pre},T}}{\sqrt{\frac{\sigma_{\text{pre},T}^2 + \sigma_{\text{post},T}^2}{2}}},
\end{equation} 
where $M_{t,T}$ is the mean of clinical scale taken at time $t$ (pre-test or post-test) and for treatment $T$ and $\sigma$ similarly represents its standard deviation.

The \gls{es} computed in this analysis was different from the one 
used previously for the replication and updating of \citet{Cortese2016}. Indeed, here we focused on the effect of the treatment within 
a group as defined by \citet{Cohen1988}, definition of the \gls{es} that was already used in the literature \citep{Arns2009, Maurizio2014, 
Strehl2017}. This \gls{es} enables to quantify the efficacy of \gls{nfb} inside the treatment group 

The \gls{es} was then considered as a dependent variable to be explained by the factors identified using the following three methods, which were 
implemented in the Scikit-Learn Python \citep[0.18.1]{Pedregosa2011} and the Statsmodels Python \citep[0.8.0]{Seabold2010} libraries:
\begin{itemize}
	\item weighted multiple linear regression (\gls{wls}) \citep{Montgomery2012}; 
	\item sparsity-regularized linear regression with \gls{lasso} \citep{Tibshirani1996};
	\item decision tree \citep{Quinlan1986}.
\end{itemize}

The most often used linear regression analysis is the \gls{ols} but here we applied the \gls{wls}: a 
weight was assigned to each observation in order to take into account the fact that some observations came from the same study because studies 
may provide several scales. Besides, the weight was a function of the sample size as well: because of their different sample sizes,
studies were not equivalent and should be analyzed accordingly. That's why the weight corresponded to the ratio between the experiment group's sample size of the study and 
the number of behavioral scales available in the study. We also ran the analysis with \gls{ols} method to assess the impact of the weights on the results. 

The aim of the \gls{wls} is to estimate the regression coefficients. A significant coefficient (meaning significantly different from 0) indicates 
that the associated factor has probably an influence on \gls{nfb} efficacy and the sign of the coefficient indicates the direction of the effect.

The second method applied was the \gls{lasso}, that naturally incorporates variable selection 
in the linear model thanks to $\ell-1$-norm applied on the coefficients. A coefficient not set at zero means that 
the associated factor may have an influence on \gls{nfb} and once again, the sign of the coefficient indicates the direction of the effect.

Eventually, the last method used to determine factors influencing \gls{nfb} was the decision tree \citep{Quinlan1986}, a non linear method. It brakes down a dataset into smaller
and smaller subsets using at each iteration a variable and a threshold chosen to optimize a simple \gls{mse} criteria \citep{James2013}. A tree is composed of several 
nodes and leafs, the importance of which is decreasing from the top node called root node. 

These methods are intrinsically different from each others, so we compared their results. They are more precisely described in the Supplemental Material.

















% adding the line below for Multifile document support with LatexTools Sublime package 
%!TEX root = manuscript.tex


% Results

\section{Results}

\subsection{Systematic review}

Search terms entered in Pubmed returned 152 results during the last check on December 14, 2017 and 28 articles included in previous
meta-analysis on \gls{nfb} were identified. After the selection process illustrated in \cref{Figure:systematic_review_workflow}, 
31 studies were included in the \gls{saob} and 15 in the meta-analysis as summarized in \cref{Table:table_factors_analysis_meta_analysis_list_studies}.
The 31 studies selected for the \gls{saob} correspond to \citeauthor{Cortese2016}'s criteria to the exception of a need for a control group since 
we are computing the within subject \gls{es}.  

\begin{figure}[h!]
  \centering
  \includegraphics[width=1.0\linewidth]{figures/meta_review_factors_analysis_how_studies_are_included_no_colors_2-columns_fitting_ima}
  \caption{Flow diagram of selection of studies (last search on December 14, 2017). The forlast subset of study exactly corresponds to the 
	studies included in \citet{Cortese2016} and more recent work meeting the same criteria. 
	The last subset ($k$=31) corresponds to the same criteria without the requirement for the presence of a control group.}
  \label{Figure:systematic_review_workflow}
\end{figure}

\begin{table}[h!]
  \centering
  \caption{List of all studies included in the three different analysis. $^a$ Studies originally included in \citet{Cortese2016}
	(search on August 30, 2015), $^b$ studies satisfying \citet{Cortese2016}'s criteria (search on December 14, 2017), $^c$ studies 
	satisfying \citet{Cortese2016}'s criteria to the exception of the part relative to the control group (search on December 14, 2017).}
  \fontsize{9}{11}\selectfont
\begin{tabular}{ cccccc }
\toprule
\multicolumn{3}{ c }{Analysis} & Study & Year & \shortstack{ Size of the \\ Neurofeedback group } \\
\midrule
 & & & \citeauthor{Arnold2014} & 2014 & 26 \\ 
 & & & \citeauthor{Bakhshayesh2011} & 2011 & 18 \\
 & & & \citeauthor{Beauregard2006} & 2006 & 15 \\
 & & & \citeauthor{Bink2014} & 2014 & 45 \\
 & & & \citeauthor{Christiansen2014} & 2014 & 14 \\
 & & & \citeauthor{Gevensleben2009} & 2009 & 59 \\
 & & & \citeauthor{Heinrich2004} & 2004 & 13 \\
 & & & \citeauthor{Holtmann2009} & 2009 & 20 \\
 & & & \citeauthor{Linden1996} & 1996 & 9 \\
 & & & \citeauthor{Maurizio2014} & 2014 & 13 \\
 & & & \citeauthor{Steiner2011} & 2011 & 9 \\
 & & & \citeauthor{Steiner2014} & 2014 & 34 \\
 & & & \citeauthor{VanDongen2013} & 2013 & 22 \\
 & & \shortstack{a = Replicate \\ \citeauthor{Cortese2016} } & 13 studies & & 297 \\
\cmidrule(lr){3-6}
 & & & \citeauthor{Baumeister2016} & 2016 & 8 \\
 & & & \citeauthor{Strehl2017} & 2017 & 72 \\
 & \shortstack{b = Update \\ \citeauthor{Cortese2016} } & & 15 studies & & 377 \\
\cmidrule(lr){2-6}
 & & & \citeauthor{Bluschke2016} & 2016 & 19 \\
 & & & \citeauthor{Deilami2016} & 2016 & 12 \\
 & & & \citeauthor{Drechsler2007} & 2007 & 17 \\
 & & & \citeauthor{Duric2012} & 2012 & 23 \\
 & & &\citeauthor{Escolano2014} & 2014 & 20 \\
 & & & \citeauthor{Fuchs2003} & 2003 & 22 \\
 & & & \citeauthor{Gelade2016} & 2016 & 39 \\
 & & & \citeauthor{Kropotov2005} & 2005 & 86 \\
 & & & \citeauthor{Lee2017} & 2017 & 18 \\
 & & & \citeauthor{Leins2007} & 2007 & 19 \\
 & & & \citeauthor{Li2013} & 2013 & 32 \\
 & & & \citeauthor{Meisel2014} & 2014 & 12 \\
 & & & \citeauthor{Mohagheghi2017} & 2017 & 30 \\
 & & & \citeauthor{Mohammadi2015} & 2015 & 16 \\
 & & & \citeauthor{Monastra2002} & 2002 & 51 \\
 & & & \citeauthor{Ogrim2013} & 2013 & 13 \\
 & & & \citeauthor{Strehl2006} & 2006 & 23 \\
 \shortstack{c = Systematic Analysis \\ of bias (\gls{saob})} & & & 32 studies & & 829 \\
\bottomrule
\end{tabular}

  \label{Table:table_factors_analysis_meta_analysis_list_studies}
\end{table}

\subsection{Perform the meta-analysis}

The Python module created in order to perform a meta-analysis was successfully validated as describes in the Supplemental Materials and made available online.
So all the following results were computed with the Python code. 

The replication and the following update of \citeauthor{Cortese2016} was conducted by applying the following choices and the results obtained are presented 
in \cref{Table:meta_review_comparison_revman_and_python_with_choices}:

\begin{itemize}
    \item to compute the \gls{es} for \citet{Arnold2014}, \citet{Cortese2016} took as post-test values the assessments after 12 \gls{nfb} sessions
		because results at post-test were not available. In our case, we use the values at post-test (i.e after the 40 sessions). With these values, 
		we find smaller \gls{es} than \citet{Cortese2016};  
    \item different results for teachers' assessments are found for \cite{Steiner2014} because we decided not to use the same scale 
		as \citeauthor{Cortese2016}. Indeed, \citeauthor{Cortese2016} relied on the BOSS Classroom Observation \citep{Shapiro2010} to compute \gls{es}
		for teachers' ratings even if this scale is not as used as other scales provided in this study. That's why we decided to conduct our analysis
		with a more common scale which has been part of studies assessing the pros and cons of different ADHD scales \citep{Epstein2012, Collett2003}: The Conners. 
		Besides, this scale is already used in this study to compute the \gls{es} for the parents' ratings. 
		Using this scale leads to higher \gls{es} in attention but not in total and hyperactivity. Moreover, this different choice of 
		scale does not affect the significance of the \glspl{es}.
\end{itemize}

\begin{table}[h!]
  \centering
  \caption{Comparison between \citet{Cortese2016} results obtained with RevMan \citep{RevMan} and those obtained with the Python code with our 
	choices applied. \glspl{se} and their corresponding p-value (in parenthesis) are presented. With the Python program, a negative \gls{se}
	is in favor of \gls{nfb}.}
\begin{tabular}{ |p{2cm}|p{2.5cm}|p{4cm}|p{4.5cm}|  }
\hline
\multicolumn{2}{ |c| }{Input data} & Results from \citet{Cortese2016} & Means and standard deviations from articles included in \citet{Cortese2016}\\
\hline
\multicolumn{2}{ |c| }{Implementation} & RevMan \citet{RevMan} & Python program\\
\hline
\multicolumn{2}{ |c| }{Hypothesis} & Same as \citet{Cortese2016} & Our choices\\
\hline
\multirow{ 3}{*}{ \textit{Parents} } & Total & $0.35$ ($0.004$) & $-0.32$ ($0.013$)\\
 & Inattention  & $0.36$ ($0.009$) & $-0.31$ ($0.036$)\\
 & Hyperactivity  & $0.26$ ($0.004$) & $-0.24$ ($0.02$)\\
\hline
\multirow{ 3}{*}{ \textit{Teachers} } & Total & $0.15$ ($0.20$) & $-0.11$ ($0.37$)\\
 & Inattention  & $0.06$ ($0.70$) & $-0.17$ ($0.16$)\\
 & Hyperactivity  & $0.17$ ($0.13$) & $-0.022$ ($0.85$)\\
 \hline
\end{tabular}

  \label{Table:meta_review_comparison_revman_and_python_with_choices}
\end{table}

Consequently, the rest of this work was carried with the following choices:
\begin{itemize}
    \item values used to compute \gls{es} from \citeauthor{Arnold2014} correspond to the scores at post-test (after the 40 sessions of \gls{nfb});  
    \item to compute the \gls{es} based on teachers' assessments, we used the Conners because this scale is more common.
\end{itemize}

The next step consists in extend \citeauthor{Cortese2016} meta-analysis by adding the two new articles \citep{Strehl2017, Baumeister2016} found 
during the systematic review. \citet{Baumeister2016} provides results only for parents total outcome whereas \citet{Strehl2017} gives teachers 
and parents' assessments for all outcomes. In spite of favorable results for \gls{nfb}, particularly on parents' assessments, adding these two 
new studies does not change either the magnitude or the significance of the \gls{se} for any outcome whatever the raters
as illustrated in \cref{Figure:meta_review_forest_plots_update_meta_analysis_our_choices_no_colors_2-columns_fitting_image}. 
 
\begin{figure}[h!]
  \centering
  \includegraphics[width=1.0\linewidth]{figures/meta_review_forest_plots_update_meta_analysis_our_choices_no_colors_2-columns_fitting_image}
  \caption{Forest plots obtained on the dataset "Update meta-analysis" with the Python code. A negative \gls{es} is in favor of \gls{nfb}. 
	The blue squares correspond to the \gls{es}, the blue diamond to the \gls{se} and the green line to the confidence interval.}
  \label{Figure:meta_review_forest_plots_update_meta_analysis_our_choices_no_colors_2-columns_fitting_image}
\end{figure}

Next, we run the analysis on two specific subgroups: on the one hand only studies following standard protocol defined by \citet{Arns2014}
are selected and on the other hand only studies forbidding participants to take medication during the clinical trial are included. 

Regarding the standard protocol subgroup, \citet{Cortese2016} found all the outcomes significant except for the hyperactivity symptoms 
rated by teachers. However, when adding \citep{Strehl2017} results, we find no significance for the inattention symptoms assessed by 
teachers as well (p-value = 0.11). 
As for the low/no medication subgroup, \glspl{se} are not significant except for the inattention symptoms assessed by parents (p-value = 0.013). 
Besides, the differences in \citet{Arnold2014} values causes a loss of significance in 
hyperactivity outcome for parents (p-value = 0.066) compared to \citet{Cortese2016} (p-value = 0.016). The two new studies are not 
included in this subgroup because subjects are taking psychostimulants during the trial.

All the scales used to compute the \gls{es} following our choices are summarized in the Supplemental Materials.

\subsection{Detect factors influencing the Neurofeedback}

This analysis is performed on 31 trials assessing the efficacy of \gls{nfb} as presented in \cref{Table:table_factors_analysis_meta_analysis_list_studies}. 
Among the 25 factors selected, 6 are removed because there are either too many missing observations or they were too homogeneous: beta up frontal, 
the use of a transfer card, the type of threshold for the rewards (incremental or fixed), the EEG quality equal to 3 and presence of a control group. 

The \gls{es} within subjects is computed for all available clinical scales of each study and then factors are associated with the computed \gls{es}
thanks to three different methods: the \gls{wls}, the \gls{lasso} and the decision tree. The global results are presented in \cref{Table:table_factors_analysis_results_summary}
where we observe that the techniques used do not return exactly the same factors. Thus, the more methods identified a factor, the more confident we can be in
the results.  

\begin{table}[h!]
  \centering
  \caption{Results of the \gls{wls}, \gls{lasso} and decision tree. For the \gls{wls}, a p-value $<$ 0.05 (in bold) means that the coefficient of 
	the corresponding factor is significantly different from 0. For the \gls{lasso}, factors not set to 0 (in bold) are selected. 
	When the value of the coefficient is negative, the corresponding factor may lead to better \gls{nfb} results.}
  \begin{center}
\begin{tabular}{ |p{4cm}|p{4cm}|p{4cm}|p{3cm}|}
\hline
\multicolumn{2}{ |c| }{Independent variables (factors)} & Influence on the \gls{nfb} \\
\hline
\multirow{ 6}{*}{ \textit{Methodological} } & age max & 0 \\
& age min & + \\
& \textbf{\gls{pblind}} & - - -  \\ 
& on drugs & - \\
& randomization & + \\  
& \gls{irb} & ++ \\  
\hline
\multirow{ 12}{*}{ \textit{Technical} } & number of sessions  & 0 \\
& session length & - \\
& \textbf{treatment length} & - - - \\
& session pace & ++ \\ 
& \gls{smr} & - - \\
& beta up central & 0 \\  
& theta down & ++ \\
& \gls{scp} & - \\ 
& transfer phase & - - \\
& \gls{eog} correction & 0 \\ 
& artifact correction based on amplitude & - - \\ 
\hline
\multirow{ 2}{*}{ \textit{Quality of acquisition} } & more than one active electrode & 0 \\ 
& \textbf{\gls{eeg} quality 2} & +++ \\  
\hline
\end{tabular}
\end{center}

  \label{Table:table_factors_analysis_results_summary}
\end{table}

First, after applying the \gls{wls}, we find that 8 factors are significantly different from zero for an adjusted R-squared of 0.74. 
When applying the \gls{ols}, the same factors are returned as significant except the transfer phase, the protocol theta down and the artifact correction
based on amplitude with an adjusted R-squared of 0.42. The \gls{lasso} regression selects significant factors by setting to 0 the others: 12 factors 
are different from 0 here. With these two methods, a negative coefficient means that the factor is in favor of the \gls{nfb}.

Eventually, the decision tree presented in \cref{Figure:factors_analysis_decision_tree_results} splits the dataset based on the factor leading to the
smallest \gls{mse}. The best predictor is the one at the top of the tree: in our case it is the \gls{pblind}. Five other factors also splits the subsets. 

\begin{figure}[h!]
  \centering
  \includegraphics[width=1.0\linewidth]{figures/factors_analysis_decision_tree_results_no_colors_2-columns_fitting_image}
  \caption{Decision Tree obtained with the factors. The criteria to minimize was the \gls{mse}. For the dummy variables, a value of 1 means
	that the factor is observed in the study. The term value corresponds to the dependent variable.}
  \label{Figure:factors_analysis_decision_tree_results}
\end{figure}

We can notice that several factors are common to the three methods, in particular the treatment length, the assessment 
by a blind rater and \gls{eeg} quality of 2 which are returned by the three methods. Besides, 
the methods agree on the direction of the influence of these factors. However, it is more difficult to interpret the influence of the factors 
returned by only one or two methods. For instance, both \gls{wls} and \gls{lasso} find that  
relying on the amplitude of the signal to correct artifacts, and including a transfer phase seem not to improve the \gls{adhd} symptoms. 
Conversely, the \gls{irb} approval, a theta down protocol, and a high number of sessions per week appear to 
positively influence the results. The decision tree and \gls{lasso} have in common the protocol \gls{smr}: it is associated with lower \gls{es}.
Five factors are returned by only one of the methods: the minimal age of the children, being on drugs 
during \gls{nfb} treatment, randomizing the groups and the \glspl{scp} protocol. 



%A negative coefficient means that the factor is in favor of the \gls{nfb}. Here, among the factors whose coefficient is significantly 
%different from 0, a long treatment, a blind rater, including a transfer phase, and correcting artifact thanks to the amplitude appear 
%to have a negative influence on the \gls{nfb} performance. Conversely, an \gls{irb} approval, a high number of sessions per week, a theta
%down protocol and an EEG quality of 2 seem to lead to more efficacy. 

%With this method, a negative coefficient corresponds also to favorable factors. Thus, having several sessions per week, being approved 
%by an \gls{irb}, a protocol theta down, and an EEG quality of 2 seem to improve the results of the \gls{nfb}. 
%On the contrary, a blind rater, a long treatment, randomizing the groups, a \glspl{scp} and \gls{smr} protocols, being on drugs during the trial, correcting 
%artifact based on the amplitude of the signal, and including a transfer phase during the session appear to be factors with a negative influence.

%So, the dataset is divided in two subsets: 
%43 samples where the assessments were reported by non-blind raters and 19 samples based on blind ratings. This last subset, where the \gls{se} is smaller 
%than the one obtained for non-blind raters, is split on the age min criteria: it appeares that the lower the age, the better the results. Regarding 
%the side of the tree where the raters are non-blind, the next factor leading to the smallest \gls{mse} is the treatment length. A smaller treatment
 %length seems to lead to better \gls{nfb} results. In that case, the EEG quality 2 enables to break down the subset and according to these results
 %if this factor is respected in studies it seems to lead to higher \gls{se}. The lower we get into the tree, the less samples are available, so results 
%were less and less reliable.

 

% words number = 1635

% adding the line below for Multifile document support with LatexTools Sublime package 
%!TEX root = manuscript.tex


% Discussion

\section{Discussion}

\subsection{Meta-analysis} 

\textcolor{red}{This replication and update of a meta-analysis did not meet all PRISMA recommendations \citep{Moher2009}. In particular, the risk of bias
in individual and across studies was not assessed.}  

In the meta-analysis performed here, we challenged some choices made by \citeauthor{Cortese2016}, which proved controversial: 
the computation of between-\gls{es} based on an unusual scale \citep{Steiner2014} and the inclusion of a pilot study \citep{Arnold2014} 
whose endpoint values were not available at the time \citeauthor{Cortese2016} conducted their meta-analysis. We here review the 
list of changes, their justification, and their impact on the analysis.
 
First, relying on the Conners-3 \citep{Conners2011} instead of the BOSS Classroom Observation \citep{Shapiro2010} for
teachers' ratings seems preferable because this scale is more commonly used \citep{Christiansen2014, Bluschke2016} and is
a revision of the Conners Rating Scale Revised \citep{Conners1998} whose reliability has been studied \citep{Collett2003}. 
However, relying on one or the other scale did not change the significance of the between-\gls{es}, regardless of outcome.

Second, to compute the between-\gls{es} of \citet{Arnold2014} the clinical scores taken when all sessions were completed were 
used instead of looking at interim results as with \citeauthor{Cortese2016}. Some studies suggested that the number of sessions 
correlates positively with the changes observed in the \gls{eeg} \citep{Vernon2004} so that a lower number of sessions would 
lead to artificially smaller between-\gls{es}. Here, the between-\gls{es} computed with the values at post test of \citet{Arnold2014} were smaller 
than those obtained after 12 sessions; however, these differences did not lead to a change of significance of the \gls{se}. 

To conclude on this meta-analysis, although some points were controversial, the impact on the
meta-analysis was minimal and did not change the statistical significance of any outcome. 
The addition of the two new studies \citep{Strehl2017, Baumeister2016} further confirmed the original results. Indeed, the
significance did not change for any outcome: the \gls{se} remained significant for \gls{mprox} raters and
non-significant for \gls{pblind}. Adding two more studies increased the significance of the sensitivity analysis run by
\citeauthor{Cortese2016}, most notably the \gls{se} of studies corresponding to \gls{nfb} "standard protocols" \citep{Arns2014}. 
While \citeauthor{Cortese2016} found that this subset tends to perform better, particularly on the \gls{pblind} outcome, 
adding two studies also extended this result to the total clinical score as well (p-value < 0.05). Despite the obvious heterogeneity 
of the studies included in this subset (particularly in terms of protocol used), these results suggest a positive relation 
between the features of this \emph{standard} design and \gls{nfb} performance. \textcolor{red}{This result is a breakthrough in the demonstration 
of standard \gls{nfb} protocol efficacy for the treatment of \gls{adhd}. Nonetheless, the  studies 
included in this subset are still highly heterogeneous (particularly in terms of protocol used), a factor which should be accounted for.}


\subsection{Factors influencing neurofeedback}

Description and analysis of different types of \gls{nfb} implementation were subject to several studies \citep{Arns2014, 
Enriquez2017, Vernon2004, Jeunet2018}. However, to the best of our knowledge none used statistical tools to quantify their influence on
clinical endpoints. 

Surprisingly, the number of sessions was not found to be significant by any method, which was somewhat
in contradiction with existing literature. For instance, \citet{Arns2014} stated that performing less than
20 \gls{nfb} sessions leads to smaller effects. Similarly, \citet{Vernon2004} observed that positive changes in the \gls{eeg}
and behavioral performance occurred after a minimum of 20 sessions. However, \citet{Enriquez2017} insisted that the number of
sessions should be carefully chosen in order to avoid "overtraining". The fact that the number of sessions was not identified as a 
positively contributing factor might be explained by the presence of only one data point with 20 sessions or less. Conceivably, 
the temporal threshold of efficacy was passed for all included studies, making the identification of this factor unlikely on 
this dataset. However, regardless of its statistical significance, the coefficient found by the \gls{wls} was negative, meaning 
that, as expected, the more sessions performed, the more efficient the \gls{nfb} tends to be. 

Interestingly, \citep{Minder2018} suggests that the subject location of the \gls{nfb} training may also be an important contributory 
factor to clinical effectiveness. However, this has been challenged by a recent study \citep{Minder2018} showing that 
performing \gls{nfb} at school or at the clinic has no significant impact on treatment response. 

The type of \gls{nfb} protocol was not identified by more than one method, and did not appear to influence the \gls{nfb} results. 
This minimal importance granted by the methods to the \gls{nfb} protocols is counter-intuitive given the centrality of the 
protocols in the neurophysiological mode of action and subsequent expected impact on
therapeutic effectiveness \citep{Vernon2004}. A possible explanation for this result is that these protocols were equally 
efficacious for the populations to whom they were offered and thereby did not constitute a significant explanatory factor. 
This result, however, does not preclude a combined and personalized strategy (offering personalized protocols based on phenotypes)
to further improve performance, as previously suggested by \citet{Alkoby2017}.

Several factors were selected by all three methods with the same direction of influence: the EEG quality, the treatment 
length, and the rater's probably blindness to the treatment. First, our analysis highlighted that recording \gls{eeg} 
in good conditions leads to better results.
This can be explained by the fact that better signal quality enables more accurate extraction of \gls{eeg} patterns
linked to \gls{adhd} and hence leads to better learning and therapeutic efficacy \citep{Congedo2004}. 
However, it remains difficult to assess the quality of \gls{eeg} hardware (such as the amplifier used) 
because little information is provided in these studies.  
This calls for greater care in future studies, which should strive to assess and report the quality of the data.

Next, it appears that the longer the course of treatment, the less efficient it becomes. Arguably, the treatment length is a
proxy for treatment intensity, suggesting that a shorter period of treatment is more likely to succeed because the frequency of the sessions
is higher. This hypothesis is supported by the fact that the variable \emph{session pace} (number of
sessions per week) is also associated with larger within-\gls{es} according to the \gls{wls} and \gls{lasso}. The impact of the
intensity of treatment has been investigated by \citep{Rogala2016} on healthy adults: it was observed that studies with
at least four training sessions completed on consecutive days were all beneficial. Overall, these results suggest adopting a high session pace, 
which is not common knowledge in the field.

\textcolor{red}{In general our results strongly support the effectiveness of \gls{nfb} for the treatment of \gls{adhd}. However, as expected, the assessment of symptoms by non-blind raters 
leads to far more favorable results than by \gls{pblind} raters, a result widely expected and in close compliance with the existing 
meta-analysis \citep{Cortese2016, Micoulaud2014}. This observation would certainly be contradictory should teachers’ 
assessments reflect a placebo effect, which has long been documented in the literature \citep{Sollie2013, Narad2015, Minder2018}. 
This point is investigated in greater detail in the following section.}

\subsection{Analysis on the probably blind raters}

Teachers were considered as \gls{pblind} raters by \citeauthor{Cortese2016} and \citeauthor{Micoulaud2014}.
Unexpectedly, the data provided did not exactly match the widely accepted hypothesis stating that the difference between
\gls{mprox} and \gls{pblind} can solely be explained by the placebo effect. 
Nonetheless, the emphasis put on 'probably' indicates that teachers may be aware of the treatment followed. 
An element that corroborates this hypothesis is the fact that, for all the studies included in this work, the amplitude 
of the clinical scale at baseline suggests that teachers did not capture the full extent of the symptoms or, put differently, 
that they were blind more to the symptoms than to the intervention, as illustrated 
in Figure~\ref{Figure:discussion_on_placebo_effect_colors_2-columns_fitting_image}. 
\textcolor{red}{Indeed, before the intervention, teachers rated the symptoms less severely compared to parents and observed less improvement at post-test: 
this tends to correspond more to case A with no placebo effect than case B}. The expected differences of ratings between 
teachers and parents have been extensively studied \citep{Sollie2013, Narad2015, Minder2018}, observing that teachers are more 
likely to underrate a child's symptom severity, especially for younger children. As a consequence, teachers might simply be less likely 
to observe a clinical change over the course of the treatment \citep{Sollie2013, Narad2015, Minder2018}. Moreover, it is also clear 
that there is more variability in teachers' scores compared to parents', which could partly explain the lower \gls{es} obtained for 
\gls{pblind} raters, since the variability deflates the \gls{es}. In conclusion, using \gls{pblind} as an estimate for correcting the 
placebo effect does not appear an appropriate choice. 

Another way to highlight a possible placebo effect is to focus on the decision tree illustrated in Figure
~\ref{Figure:factors_analysis_decision_tree_results}.
The top node splits: on the one hand 45 observations corresponding to \gls{mprox} raters and, on the other, 
21 observations corresponding to \gls{pblind}. If the differences observed between \gls{pblind} and \gls{mprox} raters were 
due to the placebo effect, one would expect to find in the \gls{mprox} sub-tree some factors linked to the perception
of the implication in the treatment. This was indeed the case: session and treatment length were found to be significant but not in the
direction corroborating the hypothesis that they are a part of a placebo effect. Indeed, one would expect that the
longer the session and the treatment, the higher the placebo effect and the greater the within-\gls{es}. Instead, the opposite was found, 
somewhat invalidating the hypothesis. 

Overall, these results suggest that \gls{pblind} assessments could hardly be used to assess placebo effect as they seem to be blinder 
to symptoms than to intervention. In the absence of an ethically \citep{Holtmann2014} and technically \citep{Birbaumer1991} feasible sham 
for \gls{nfb} protocols \citep{World-Medical-Association2000}, it is necessary to fall back on an acceptable methodological alternative for 
the demonstration of clinical effectiveness. Among those are the analysis of neuromarkers collected during \gls{nfb} treatment demonstrating 
that patients do \emph{control} the trained neuromarkers; that they \emph{learn} (reinforce control over time), and that these possibly 
lead to lasting brain reorganization (e.g., changes in their baseline resting state activity). The specificity of these changes, in relation 
to which neuromarkers were trained and to the clinical improvement, will be an essential component of this demonstration.  

% words number = 1506

% adding the line below for Multifile document support with LatexTools Sublime package 
%!TEX root = manuscript.tex

% conclusion

\section{Conclusion}

%This work confirms \citet{Cortese2016}'s findings in the light of recently published clinical works.
%In particular, studies following a standard protocol as defined by \citet{Arns2014} show significant 
%clinical improvements on \gls{pblind} raters (k = 4 studies instead of 3 \citet{Cortese2016}).

%Besides a meta-analysis, a new method is suggested here to tackle the high heterogeneity of clinical data 
%available on \gls{nfb}. This method aims at identifying factors that are positively or negatively contributing 
%to \gls{nfb} efficacy. Three factors were consistently found to explain clinical within-\gls{es}. First, the 
%quality of acquisition of the \gls{eeg} was positively correlated with clinical efficacy. This supports a mode of 
%action through specific \gls{eeg} training. Likewise, treatment intensity was found to contribute, corroborating what 
%is known from learning theory (memory consolidation) \citep{Mowrer1960}, that is, a more intense treatment leads to 
%an increased clinical efficacy. Finally, results show that the therapeutic efficacy measured by teachers is reduced 
%compared to that measured by parents. This result has long been documented and it is widely advanced that this 
%difference is solely imputable to the amplitude of placebo effect in \gls{nfb}. However, the data presented in this article, 
%in line with the most recent work on the topic \citep{Sollie2013, Narad2015, Minder2018} 
%tends to refute this hypothesis and suggests instead that teachers are simply less likely to be exposed to symptoms. 
%As a consequence, using \gls{pblind} endpoints to address the specificity of the clinical efficacy is not recommended 
%and we advice instead to rely on other available methodological tools. \textcolor{red}{Those include sham \gls{nfb} and neuromarker 
%analysis investigating the specificity of the \gls{eeg} changes with respect to trained neuromarkers as well as changes 
%in clinical endpoints.}
%
%These elements converge to the conclusion that existing methodologies, in particular the double blind design, 
%traditionally used to assess pharmacological treatments in neuropsychiatric conditions may not be fully fitted to the evaluation 
%of medical devices. The series of results presented here, however, suggest the presence of a genuine signal in favor of 
%the therapeutic efficacy of \gls{nfb}. A signal that should nonetheless be studied further using the aforementioned methodological 
%tools, neuromarker analysis in the first place.


\textcolor{red}{In this work we provide additional elements in favor of the effectiveness of \gls{nfb} for the treatment of \gls{adhd}. First, 
we confirm that a subgroup of standard \gls{nfb} studies show a statistically significant improvement on \gls{pblind} 
assessments (k = 4 studies instead of 3, n = 283 patients instead of 158} \citet{Cortese2016}). 

\textcolor{red}{Second, we identify technical factors as positive contributors to clinical effectiveness, which strongly suggests 
that it is mediated by a real mechanism of action based on \gls{eeg} conditioning. Equally, treatment intensity was also found to 
contribute, corroborating what is known from learning theory (memory consolidation)} \citep{Mowrer1960}\textcolor{red}{; that is to say, a more intense treatment leads to 
an increased clinical efficacy.}

\textcolor{red}{While these findings certainly contribute to the debate, this work also suggests that the ultimate demonstration of evidence 
remains out of reach, as teachers’ assessments were partly invalidated as a proxy for the quantification of the placebo effect. 
As a consequence, using \gls{pblind} endpoints to address the specificity of the clinical efficacy is not recommended 
and we instead advise a reliance on other available methodological tools. These tools include sham \gls{nfb} and neuromarker 
analysis investigating the specificity of the \gls{eeg} changes with respect to trained neuromarkers as well as changes 
in clinical endpoints.}

\textcolor{red}{This work also offers an open-source toolbox for running meta-analysis and \gls{saob}: the code and data used are available, 
thus ensuring the transparency and replicability of these analysis, as well as fostering future ones.}

% 375

% discolsure statement 

\section*{Conflict of Interest Statement}
A. Bussalb, Q. Barthelemy, D. Ojeda, and L. Mayaud work for Mensia Technologies.
M. Congedo served as an advisor for Mensia Technologies when this work was conducted.  

\section*{Author Contributions}
AB extracted all data from articles and performed the analysis, MC provided useful advice 
concerning statistics and the methods used, RD and EA followed closely the evolution of 
the work and provided good ideas especially on the clinical level. QB and DO gave useful 
ideas to optimally apply methods used in the \gls{saob} and helped in the interpretation 
of the results. LM oversaw all this work. 

\section*{Funding}
This research was conducted as part of a PhD thesis funded by Mensia Technologies, the 
Association Nationale Recherche et Technologie (ANRT), and the EU H2020 NEWROFEED grant.


% ACKNOWLEDGEMENTS

\section{Acknowledgments}

We would like to thank Quentin Barthelemy, PhD and David Ojeda, PhD for their helpful comments and ideas on that work. 

\clearpage

\bibliography{bibliography}

\clearpage

\section*{Figure captions}

\begin{figure}[h!]
  \centering
	\includegraphics[width=1.0\linewidth]{figures/meta_review_factors_analysis_how_studies_are_included_no_colors_2-columns_fitting_ima} 
  \caption{Flow diagram of selection of studies (last search on December 14, 2017).  
	The subset (a) corresponds to the \citeauthor{Cortese2016}'s inclusion criteria without the requirement for the presence of a control group. The number
	of patients is only equal to the subjects in the \gls{nfb} groups.
	The subset (b) exactly corresponds to the studies included in \citet{Cortese2016} and more recent work meeting the same criteria. Here, the number of patients includes all patients
	whatever their treatment group.}
  \label{Figure:systematic_review_workflow}
\end{figure}

\begin{figure}[h!]
  \centering
  \includegraphics[width=1.0\linewidth]{figures/meta_review_forest_plots_update_meta_analysis_our_choices_no_colors_2-columns_fitting_image}
  \caption{Forest plots obtained on the dataset "Update meta-analysis" with the Python code. The \gls{es} presented here correspond to
	the between subject effect size. A negative \gls{es} is in favor of \gls{nfb}. 
	The blue squares correspond to the \gls{es}, the blue diamond to the \gls{se} and the green line to the 95\% confidence interval.}
  \label{Figure:meta_review_forest_plots_update_meta_analysis_our_choices_no_colors_2-columns_fitting_image}
\end{figure}

\begin{figure}[h!]
  \centering
  \includegraphics[width=1.0\linewidth]{figures/factors_analysis_decision_tree_results_no_colors_2-columns_fitting_image}
  \caption{Decision Tree obtained: \gls{es} corresponds to the within subject effect size and k to the number of studies. 
	The criteria to minimize was the \gls{mse}. The importance of nodes and leafs is decreasing
	from the root node.}
  \label{Figure:factors_analysis_decision_tree_results}
\end{figure}

\begin{figure}[h!]
  \centering
  \includegraphics[width=1.0\linewidth]{figures/discussion_on_placebo_effect_colors_2-columns_fitting_image}
  \caption{Pre-test and post-test scores ($\pm$ standard error) given by Parents (\gls{mprox}) in blue and teachers (\gls{pblind}) in green. Two configurations: \textbf{(A)} teachers don’t see the symptoms at 
	pre-test so they can’t see any improvement at post-test, \textbf{(B)} teachers see the symptoms at pre-test and don’t see any improvement at post-test. \textbf{(C)} Evolution of parents and teachers' scores
	between pre and post-test on studies that satisfy \citeauthor{Cortese2016}'s inclusion criteria and that provide teachers and parents scores on the same scale.}
  \label{Figure:discussion_on_placebo_effect_colors_2-columns_fitting_image}
\end{figure} 

\clearpage

\section*{Table captions}

\begin{table}[h!]
  \centering
  \caption{List of all studies included in the three different analysis. $^a$ Studies originally included in \citet{Cortese2016}
	(search on August 30, 2015), $^b$ studies satisfying \citet{Cortese2016}'s criteria (search on December 14, 2017), $^c$ studies 
	satisfying \citet{Cortese2016}'s criteria to the exception of the part relative to the control group (search on December 14, 2017).}
  \fontsize{9}{11}\selectfont
\begin{tabular}{ cccccc }
\toprule
\multicolumn{3}{ c }{Analysis} & Study & Year & \shortstack{ Size of the \\ Neurofeedback group } \\
\midrule
 & & & \citeauthor{Arnold2014} & 2014 & 26 \\ 
 & & & \citeauthor{Bakhshayesh2011} & 2011 & 18 \\
 & & & \citeauthor{Beauregard2006} & 2006 & 15 \\
 & & & \citeauthor{Bink2014} & 2014 & 45 \\
 & & & \citeauthor{Christiansen2014} & 2014 & 14 \\
 & & & \citeauthor{Gevensleben2009} & 2009 & 59 \\
 & & & \citeauthor{Heinrich2004} & 2004 & 13 \\
 & & & \citeauthor{Holtmann2009} & 2009 & 20 \\
 & & & \citeauthor{Linden1996} & 1996 & 9 \\
 & & & \citeauthor{Maurizio2014} & 2014 & 13 \\
 & & & \citeauthor{Steiner2011} & 2011 & 9 \\
 & & & \citeauthor{Steiner2014} & 2014 & 34 \\
 & & & \citeauthor{VanDongen2013} & 2013 & 22 \\
 & & \shortstack{a = Replicate \\ \citeauthor{Cortese2016} } & 13 studies & & 297 \\
\cmidrule(lr){3-6}
 & & & \citeauthor{Baumeister2016} & 2016 & 8 \\
 & & & \citeauthor{Strehl2017} & 2017 & 72 \\
 & \shortstack{b = Update \\ \citeauthor{Cortese2016} } & & 15 studies & & 377 \\
\cmidrule(lr){2-6}
 & & & \citeauthor{Bluschke2016} & 2016 & 19 \\
 & & & \citeauthor{Deilami2016} & 2016 & 12 \\
 & & & \citeauthor{Drechsler2007} & 2007 & 17 \\
 & & & \citeauthor{Duric2012} & 2012 & 23 \\
 & & &\citeauthor{Escolano2014} & 2014 & 20 \\
 & & & \citeauthor{Fuchs2003} & 2003 & 22 \\
 & & & \citeauthor{Gelade2016} & 2016 & 39 \\
 & & & \citeauthor{Kropotov2005} & 2005 & 86 \\
 & & & \citeauthor{Lee2017} & 2017 & 18 \\
 & & & \citeauthor{Leins2007} & 2007 & 19 \\
 & & & \citeauthor{Li2013} & 2013 & 32 \\
 & & & \citeauthor{Meisel2014} & 2014 & 12 \\
 & & & \citeauthor{Mohagheghi2017} & 2017 & 30 \\
 & & & \citeauthor{Mohammadi2015} & 2015 & 16 \\
 & & & \citeauthor{Monastra2002} & 2002 & 51 \\
 & & & \citeauthor{Ogrim2013} & 2013 & 13 \\
 & & & \citeauthor{Strehl2006} & 2006 & 23 \\
 \shortstack{c = Systematic Analysis \\ of bias (\gls{saob})} & & & 32 studies & & 829 \\
\bottomrule
\end{tabular}

  \label{Table:table_factors_analysis_meta_analysis_list_studies}
\end{table}

\begin{table}[h!]
  \centering
  \caption{Comparison between \citet{Cortese2016} results obtained with RevMan \citep{RevMan} and those obtained with the Python code with our 
	choices applied ($^a$ post-test values for \citeauthor{Arnold2014} are obtained after 40 sessions of \gls{nfb} and Conners scale is used for \citeauthor{Steiner2014}
	teachers' outcomes). \glspl{se} and their corresponding p-value (in parenthesis) are presented. With the Python program, a negative \gls{se}
	is in favor of \gls{nfb} unlike \citeauthor{Cortese2016}.}
\begin{tabular}{ |p{2cm}|p{2.5cm}|p{4cm}|p{4.5cm}|  }
\hline
\multicolumn{2}{ |c| }{Input data} & Results from \citet{Cortese2016} & Means and standard deviations from articles included in \citet{Cortese2016}\\
\hline
\multicolumn{2}{ |c| }{Implementation} & RevMan \citet{RevMan} & Python program\\
\hline
\multicolumn{2}{ |c| }{Hypothesis} & Same as \citet{Cortese2016} & Our choices\\
\hline
\multirow{ 3}{*}{ \textit{Parents} } & Total & $0.35$ ($0.004$) & $-0.32$ ($0.013$)\\
 & Inattention  & $0.36$ ($0.009$) & $-0.31$ ($0.036$)\\
 & Hyperactivity  & $0.26$ ($0.004$) & $-0.24$ ($0.02$)\\
\hline
\multirow{ 3}{*}{ \textit{Teachers} } & Total & $0.15$ ($0.20$) & $-0.11$ ($0.37$)\\
 & Inattention  & $0.06$ ($0.70$) & $-0.17$ ($0.16$)\\
 & Hyperactivity  & $0.17$ ($0.13$) & $-0.022$ ($0.85$)\\
 \hline
\end{tabular}

  \label{Table:meta_review_comparison_revman_and_python_with_choices}
\end{table}

\begin{table}[h!]
  \centering
  \caption{Results of the \gls{wls}, \gls{lasso} and decision tree. For the \gls{wls}, a p-value $<$ 0.05 (in bold) means that the coefficient of 
	the corresponding factor is significantly different from 0. For the \gls{lasso}, factors not set to 0 (in bold) are selected. For the decision tree,
	the place of the factor in the tree is precised. When the value of the coefficient is negative, the corresponding factor may lead to better \gls{nfb} results.}
  \begin{center}
\begin{tabular}{ |p{4cm}|p{4cm}|p{4cm}|p{3cm}|}
\hline
\multicolumn{2}{ |c| }{Independent variables (factors)} & Influence on the \gls{nfb} \\
\hline
\multirow{ 6}{*}{ \textit{Methodological} } & age max & 0 \\
& age min & + \\
& \textbf{\gls{pblind}} & - - -  \\ 
& on drugs & - \\
& randomization & + \\  
& \gls{irb} & ++ \\  
\hline
\multirow{ 12}{*}{ \textit{Technical} } & number of sessions  & 0 \\
& session length & - \\
& \textbf{treatment length} & - - - \\
& session pace & ++ \\ 
& \gls{smr} & - - \\
& beta up central & 0 \\  
& theta down & ++ \\
& \gls{scp} & - \\ 
& transfer phase & - - \\
& \gls{eog} correction & 0 \\ 
& artifact correction based on amplitude & - - \\ 
\hline
\multirow{ 2}{*}{ \textit{Quality of acquisition} } & more than one active electrode & 0 \\ 
& \textbf{\gls{eeg} quality 2} & +++ \\  
\hline
\end{tabular}
\end{center}

  \label{Table:table_factors_analysis_results_summary}
\end{table}

\end{document}