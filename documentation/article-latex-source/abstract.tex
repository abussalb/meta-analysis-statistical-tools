% adding the line below for Multifile document support with LatexTools Sublime package 
%!TEX root = manuscript.tex

% Abstract

\noindent \textcolor{red}{Meta-analysis are very often relied on to evaluate Neurofeedback (NFB) treatment for Attention Deficit 
Hyperactivity Disorder (ADHD) in children and adolescents. That’s why this work aims to replicate, while investigating 
new choices, and extent the last meta-analysis on this subject \citep{Cortese2016} by adding two randomized 
control trials (RCTs) that have been published since. By doing so, we realized that this method presents some 
loopholes: first, teachers, which are considered as probably blind, seem actually to be blinder to symptoms 
than to intervention; second, NFB literature is characterized by a high technical and methodological heterogeneity, 
which may partly explain the lack of efficacy consensus across studies. However, we can take advantage of this 
heterogeneity by performing a systematic analysis of biases (SAOB) of the studies included in the previous meta-analysis, 
rather than suffering from it. This approach aims to find, methodological as well as technical, factors that would 
influence NFB results.} Our extended meta-analysis (k = 15 studies) confirmed the results previously obtained: effect sizes 
in favor of NFB efficacy were significant when clinical scales of ADHD were rated by parents (non-blind, p-value = 0.0017), 
but not when rated by teachers (probably blind, p-value = 0.14). The effect size was significant according to both raters for 
the subset of studies meeting the definition of "standard NFB protocols" (parents p-value = 0.0054; teachers p-value = 0.043, k = 4). 
Then, the SAOB was performed on k = 31 trials, meeting the same inclusion criteria as the earlier meta-analysis (with the exception 
of the control arm). This approach identified three main factors that have an impact on NFB efficacy: first, a more intensive treatment, 
but not treatment duration, was associated with higher efficacy; second, teachers reported a lower improvement as compared to parents; 
third, high-quality EEG systems improved the effectiveness of the NFB treatment. \textcolor{red}{Even if this last technical point is in favor of the 
specificity of NFB treatment, we can’t conclude on its efficacy. To do so, a RCT is required, followed by neuromarker analysis 
investigating the specificity of the EEG changes with respect to trained neuromarkers}. 

% frontiers: 350
% 341 words
