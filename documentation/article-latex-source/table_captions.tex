\section*{Table captions}

\begin{table}[h!]
  \centering
  \caption{List of all studies included in the three different analysis. $^a$ Studies originally included in \citet{Cortese2016}
	(search on August 30, 2015), $^b$ studies satisfying \citet{Cortese2016}'s criteria (search on December 14, 2017), $^c$ studies 
	satisfying \citet{Cortese2016}'s criteria to the exception of the part relative to the control group (search on December 14, 2017).}
  \fontsize{9}{11}\selectfont
\begin{tabular}{ cccccc }
\toprule
\multicolumn{3}{ c }{Analysis} & Study & Year & \shortstack{ Size of the \\ Neurofeedback group } \\
\midrule
 & & & \citeauthor{Arnold2014} & 2014 & 26 \\ 
 & & & \citeauthor{Bakhshayesh2011} & 2011 & 18 \\
 & & & \citeauthor{Beauregard2006} & 2006 & 15 \\
 & & & \citeauthor{Bink2014} & 2014 & 45 \\
 & & & \citeauthor{Christiansen2014} & 2014 & 14 \\
 & & & \citeauthor{Gevensleben2009} & 2009 & 59 \\
 & & & \citeauthor{Heinrich2004} & 2004 & 13 \\
 & & & \citeauthor{Holtmann2009} & 2009 & 20 \\
 & & & \citeauthor{Linden1996} & 1996 & 9 \\
 & & & \citeauthor{Maurizio2014} & 2014 & 13 \\
 & & & \citeauthor{Steiner2011} & 2011 & 9 \\
 & & & \citeauthor{Steiner2014} & 2014 & 34 \\
 & & & \citeauthor{VanDongen2013} & 2013 & 22 \\
 & & \shortstack{a = Replicate \\ \citeauthor{Cortese2016} } & 13 studies & & 297 \\
\cmidrule(lr){3-6}
 & & & \citeauthor{Baumeister2016} & 2016 & 8 \\
 & & & \citeauthor{Strehl2017} & 2017 & 72 \\
 & \shortstack{b = Update \\ \citeauthor{Cortese2016} } & & 15 studies & & 377 \\
\cmidrule(lr){2-6}
 & & & \citeauthor{Bluschke2016} & 2016 & 19 \\
 & & & \citeauthor{Deilami2016} & 2016 & 12 \\
 & & & \citeauthor{Drechsler2007} & 2007 & 17 \\
 & & & \citeauthor{Duric2012} & 2012 & 23 \\
 & & &\citeauthor{Escolano2014} & 2014 & 20 \\
 & & & \citeauthor{Fuchs2003} & 2003 & 22 \\
 & & & \citeauthor{Gelade2016} & 2016 & 39 \\
 & & & \citeauthor{Kropotov2005} & 2005 & 86 \\
 & & & \citeauthor{Lee2017} & 2017 & 18 \\
 & & & \citeauthor{Leins2007} & 2007 & 19 \\
 & & & \citeauthor{Li2013} & 2013 & 32 \\
 & & & \citeauthor{Meisel2014} & 2014 & 12 \\
 & & & \citeauthor{Mohagheghi2017} & 2017 & 30 \\
 & & & \citeauthor{Mohammadi2015} & 2015 & 16 \\
 & & & \citeauthor{Monastra2002} & 2002 & 51 \\
 & & & \citeauthor{Ogrim2013} & 2013 & 13 \\
 & & & \citeauthor{Strehl2006} & 2006 & 23 \\
 \shortstack{c = Systematic Analysis \\ of bias (\gls{saob})} & & & 32 studies & & 829 \\
\bottomrule
\end{tabular}

  \label{Table:table_factors_analysis_meta_analysis_list_studies}
\end{table}

\begin{table}[h!]
  \centering
  \caption{Comparison between \citet{Cortese2016} results obtained with RevMan \citep{RevMan} and those obtained with the Python code with our 
	choices applied ($^a$ post-test values for \citeauthor{Arnold2014} are obtained after 40 sessions of \gls{nfb} and Conners scale is used for \citeauthor{Steiner2014}
	teachers' outcomes). \glspl{se} and their corresponding p-value (in parenthesis) are presented. With the Python program, a negative \gls{se}
	is in favor of \gls{nfb} unlike \citeauthor{Cortese2016}.}
\begin{tabular}{ |p{2cm}|p{2.5cm}|p{4cm}|p{4.5cm}|  }
\hline
\multicolumn{2}{ |c| }{Input data} & Results from \citet{Cortese2016} & Means and standard deviations from articles included in \citet{Cortese2016}\\
\hline
\multicolumn{2}{ |c| }{Implementation} & RevMan \citet{RevMan} & Python program\\
\hline
\multicolumn{2}{ |c| }{Hypothesis} & Same as \citet{Cortese2016} & Our choices\\
\hline
\multirow{ 3}{*}{ \textit{Parents} } & Total & $0.35$ ($0.004$) & $-0.32$ ($0.013$)\\
 & Inattention  & $0.36$ ($0.009$) & $-0.31$ ($0.036$)\\
 & Hyperactivity  & $0.26$ ($0.004$) & $-0.24$ ($0.02$)\\
\hline
\multirow{ 3}{*}{ \textit{Teachers} } & Total & $0.15$ ($0.20$) & $-0.11$ ($0.37$)\\
 & Inattention  & $0.06$ ($0.70$) & $-0.17$ ($0.16$)\\
 & Hyperactivity  & $0.17$ ($0.13$) & $-0.022$ ($0.85$)\\
 \hline
\end{tabular}

  \label{Table:meta_review_comparison_revman_and_python_with_choices}
\end{table}

\begin{table}[h!]
  \centering
  \caption{Results of the \gls{wls}, \gls{lasso} and decision tree. For the \gls{wls}, a p-value $<$ 0.05 (in bold) means that the coefficient of 
	the corresponding factor is significantly different from 0. For the \gls{lasso}, factors not set to 0 (in bold) are selected. For the decision tree,
	the place of the factor in the tree is precised. When the value of the coefficient is negative, the corresponding factor may lead to better \gls{nfb} results.}
  \begin{center}
\begin{tabular}{ |p{4cm}|p{4cm}|p{4cm}|p{3cm}|}
\hline
\multicolumn{2}{ |c| }{Independent variables (factors)} & Influence on the \gls{nfb} \\
\hline
\multirow{ 6}{*}{ \textit{Methodological} } & age max & 0 \\
& age min & + \\
& \textbf{\gls{pblind}} & - - -  \\ 
& on drugs & - \\
& randomization & + \\  
& \gls{irb} & ++ \\  
\hline
\multirow{ 12}{*}{ \textit{Technical} } & number of sessions  & 0 \\
& session length & - \\
& \textbf{treatment length} & - - - \\
& session pace & ++ \\ 
& \gls{smr} & - - \\
& beta up central & 0 \\  
& theta down & ++ \\
& \gls{scp} & - \\ 
& transfer phase & - - \\
& \gls{eog} correction & 0 \\ 
& artifact correction based on amplitude & - - \\ 
\hline
\multirow{ 2}{*}{ \textit{Quality of acquisition} } & more than one active electrode & 0 \\ 
& \textbf{\gls{eeg} quality 2} & +++ \\  
\hline
\end{tabular}
\end{center}

  \label{Table:table_factors_analysis_results_summary}
\end{table}