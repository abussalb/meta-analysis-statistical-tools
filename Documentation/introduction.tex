% adding the line below for Multifile document support with LatexTools Sublime package 
%!TEX root = manuscript.tex

% Introduction

\section{Introduction} 

\gls{adhd} is a common psychiatric disorder of childhood characterized by impaired attention and/or hyperactivity/impulsivity, 
symptoms which may persist in adulthood with clinical significance which makes \gls{adhd} a life-long problem for many patients 
\citep{Faraone2006}. The prevalence of \gls{adhd} is about 5\% in school-aged children yielding to an estimated 2.5 millions of 
children in Europe \citep{DSM-5}. \gls{adhd} has an impact on the children well being because many of them may have low self 
esteem \citep{Shaw2005} and underachieve at school \citep{Barry2002} but parents are also affected by this situation \citep{Harpin2005}: 
they are often stigmatized due to the fact that for many, the behavior of children with \gls{adhd} is solely explained by bad 
parenting. Besides \gls{adhd} has a financial cost: it is estimated at between \$12,005 and \$17,458 per individual annually \citep{Pelham2007}. 

The diagnosis of \gls{adhd} primarily relies on questionnaire-based clinical evaluation \citep{DSM-5}, which can be supported 
with objective assessment metrics of executive function such as the \gls{tova} \citep{Forbes1998}, the \gls{cpt} \citep{Barkley1991} 
and the \gls{sart} \citep{Robertson1997}. On the contrary, objective markers of brain function using \gls{eeg}, \gls{fmri}, or \gls{pet}
could not successfully improve diagnosis \citep{Neba} at the individual level but proved significantly different on the 
population. More specifically, these studies allowed to identify specific neurophysiological phenotypes of \gls{adhd}: 
this was particularly reported with \gls{eeg} recordings \citep{loo2017}. For instance, \gls{adhd} patients were found to show 
an increase in theta waves (4-8Hz) in the frontal area whereas there are less beta waves (12-32Hz) and \gls{smr} (13-15Hz) 
in the central area \citep{Monastra2005, Matouvsek1984, Janzen1995}.  
 
Among all existing treatments, the most widely used is the psychostimulants, e.g. \gls{mph}, which has been proven to be 
efficacious \citep{Taylor2014, Storebo2015}. However, the long-term effects when taking psychostimulants
are not established: it seems that the decrease of \gls{adhd} symptoms does not persist when the patient stops the treatment
\citep{DuPaul1998, Swanson2001, Jensen1999}. Moreover, \gls{adhd} children under medication commonly suffer from side effects
such as loss of appetite and sleep problems but no serious adverse events have been reported \citep{Storebo2015, Cooper2011}. 
These drawbacks make some parents and clinicians reluctant to choose such medications, so they turn to drug-free
treatment options such as dietary changes \citep{Belanger2009} and behavioral therapy which are in most of cases less efficient \citep{Sonuga-Barke2013}.

\gls{nfb} is a noninvasive technique based on behavioral therapy that aims to reduce the \gls{adhd} symptoms \citep{Arns2015, Steffert2010}.
It is a self-paced brain neuromodulation technique that represents one's brain activity in real-time using auditory or 
visual modulations, on which learning paradigms can be applied such as operant conditioning or voluntary control.
To deliver this intervention, neurophysiological time series must be recorded and analyzed in real-time and implemented in serious games 
leveraging learning paradigms. To that effect, recorded brain signals are analyzed to extract a real time representations of the activity 
of a population of neurons involved in attentional networks to which learning paradigms are applied, which is translated into a visual 
or auditory cues. The sensory feedback constitutes the rewards mechanism that promotes learning using a well-known operant conditioning protocol. 
The operant conditioning principle will enable the child to repeat more and more easily this task and thanks to the natural neuronal plasticity,
a neuronal reorganization is observed \citep{VanDoren2017}. 

In case of \gls{adhd}, several \gls{nfb} protocols have been proposed and investigated to decrease the symptoms: 
\begin{itemize}
	\item protocols based on frequency band training: a child can be asked to enhance his \gls{smr} 
	while suppressing theta or beta \citep{Lubar1976}, or he can have to enhance beta
	while suppressing theta (this scenario is known as \gls{tbr}) \citep{Arns2013};
	\item protocol based on the \glspl{scp} training which consists in the regulation of cortical excitation 
	thresholds by focusing on activity generated by external cues 
	(similar to \glspl{erp}) \citep{Heinrich2004, Banaschewski2007}; 
	\item protocol based on \glspl{erp} (P300) \citep{Fouillen2017}: \gls{adhd} children have a reduced P300 
	amplitude so it can be considered as a specific neurophysiological marker of selective attention. 
\end{itemize} 

Shortly after the discovery of the brain's electric activity by Hans Berger in 1924, \citet{Durup1935} proved it could be voluntarily modulated. 
The first indication of its therapeutic potential came forty years later when \citet{Sterman1974} serendipitously found the training of \gls{smr} 
activity to reduce the incidence of epileptic crisis in kerozen-exposed cats. The technique, then known as \gls{nfb} quickly became investigated in 
various fields of neuropsychiatry including, most notably, \gls{adhd} and resulting in a relatively large body of scientific literature 
\citep{Lubar1976, Rossiter1995, Linden1996, Maurizio2014}. Subsequently, its efficacy on the core symptoms of \gls{adhd} (inattention, hyperactivity 
and impulsivity) has been subject to several meta-analytic studies \citep{Loo2005, Lofthouse2012, Arns2009, Micoulaud2014, Sonuga-Barke2013}. 

%\comment{Not sure how, this paragraph is structured. I would take the message we carved for your SOFTAL presentation: 1. What is it? 2. Prevalence/incidence, 3. Consequences (social and financial),
% 4. diagnosis methods (introduce limitation of biomarkers), 
% 5. Existing treatment and their limitation. Then only introduce NFB with its origins. You will find interesting material and references in the CER document. }

The most recent meta-analysis solely on the efficacy of \gls{nfb} has been conducted by \citet{Cortese2016} in 
which 13 studies are included. Although only \glspl{rct} are selected, the authors of 
this meta-analysis have made some choices which have been debated by the community in particular by 
\citet{Micoulaud2016} who criticized the use of an uncommon behavioral scale provided by \citet{Steiner2014}
for the teachers' assessments and the inclusion of a pilot study carried out by \citet{Arnold2014} in the meta-analysis. 

Because of the publication of new researches meeting \citeauthor{Cortese2016}'s inclusion criteria, we decided to update his work and take 
the opportunity to investigate some choices that later proved controversial. Eventually, we extended the analysis with a \gls{saob} taking 
advantage of studies technical and methodological high heterogeneity rather than suffering from it. Indeed, the \gls{nfb} 
domain is characterized by a clinical literature that is tremendously heterogeneous: studies differ on a methodological 
point of view (randomization and presence of a blind assessor for instance), but also on the \gls{nfb} implementation (number of sessions,
session and treatment length and type of protocol for example) and on the acquisition and pre-processing of the \gls{eeg}. Since we supposed that the 
methodological and technical choices made by authors may lead to various \gls{nfb} results, we propose here to identify which of the factors 
independently influence the reported \gls{es} thanks to adequate statistical tools.


%\comment{This last paragraph is not very clear for me either even though I know what you did and why. I would clearly split your point as follow:
% 1. we wanted to replicate Cortese's work in the light of recently published clinical work meeting his inclusion/selection criteria, 2. this was a
% good opportunity to study the sensitivity of his methodological choices that were questioned, and finally 3.
%* given the large heterogeneity of the studies included in the analysis (detail the heterogeneity and give some specifics in the intro) 
%we decided to offer a new framework for the analysis so as to benefit from it - rather than it constitutiong a major limitation of the work.  }

% number of words: 821





